%%% This is the template Gessert has send us.     %%%
%%% Look at it for further explanation and tips.  %%%
%%%%%%%%%%%%%%%%%%%%%%%%%%%%%%%%%%%%%%%%%%%%%%%%%%%%%

\documentclass[runningheads,a4paper]{llncs}

\usepackage{amssymb}
\setcounter{tocdepth}{3}
\usepackage{graphicx}

\usepackage{url}
\urldef{\mailsa}\path|{felix.wege, yuria.konda, lakshamanan.chockalingam}@tuhh.de|    
\newcommand{\keywords}[1]{\par\addvspace\baselineskip
\noindent\keywordname\enspace\ignorespaces#1}

\graphicspath{{figures/}}

%%%%%%%%%%%%%%%%%%%%%%%%%%%%%%%%%%%%%%%%%%%%%%%%%%%%%
\begin{document}

\mainmatter  % start of an individual contribution

% first the title is needed
\title{Force Prediction using OCT Sensor Data 
by Linear Regression, CNN and RNN}

\author{Felix Wege \and Yuria Konda \and Lakshamanan Chockalingam}

\authorrunning{Force Estimation using OCT Sensor Data}
% (feature abused for this document to repeat the title also on left hand pages)

\institute{Intelligent Systems in Medicine, Institute of Medical Technology,\\
Am Schwarzenberg-Campus 3, 21073 Hamburg\\
\mailsa\\
}

\maketitle


\begin{abstract}
Content of this student project is to develop a model that gives 
information about the force acting at a needle tip while plunging it against 
a tissue.
The purpose is to give the operator some kind of force feedback
without measuring it explicitly.
Input of this model is an optical coherence tomography (OCT) signal
that is measured at the needle tip.
Three models which are linear regression, convolutional neural network
and recursive neural network are considered.

\keywords{Machine learning, linear regression, CNN, RNN, optical coherence tomography}
\end{abstract}

%%%%%%%%%%%%%%%%%%%%%%%%%%%%%%%%%%%%%%%%%%%%%%%%%%%%%
%%%     Section - Introduction                    %%%	
%%%%%%%%%%%%%%%%%%%%%%%%%%%%%%%%%%%%%%%%%%%%%%%%%%%%%
\section{Introduction}\label{sec:intro}
% INTRODUCTION % 

%\section{Introduction}\label{sec:introduction}


%%%%%%%%%%%%%%%%%%%%%%%%%%%%%%%%%%%%%%%%%%%%%%%%%%%%%
%%%     Section - Data Acquisition                %%%
%%%%%%%%%%%%%%%%%%%%%%%%%%%%%%%%%%%%%%%%%%%%%%%%%%%%%
\section{Data Acquisition}\label{sec:dataAcqui}
% DATA ACQUISITION%

%\Section{Data Acquisition}\label{sec:dataAcqui}

As stated in the introduction \ref{sec:intro}, models are build to estimate the force at the needle tip 
using the OCT data as input.
For optimization and supervised learning, ground truth data are necessary. For this, a force sensor is 
integrated into the OCT system by placing it at the end of the needle.

*PICTURE*

Data is collected by moving the needle forward to poke against a metal plate continuously monitored.
The transparent tip part of the needle deforms what leads to a faster reflection of the light in the OCT sensor and thus
changes the depth of the maximal reflection.
This set up only measures frontal forces and calibrates the system ue to the stiff 


%%%%%%%%%%%%%%%%%%%%%%%%%%%%%%%%%%%%%%%%%%%%%%%%%%%%%
%%%     Section - Models                          %%%
%%%%%%%%%%%%%%%%%%%%%%%%%%%%%%%%%%%%%%%%%%%%%%%%%%%%%
\section{Models}

% ToDo Introduction to models

%%%%%%%%%%%%%%%%%%%%%%%%%%%%%%%%%%%%%%%%%%%%%%%%%%%%%
%%%     Subsection - Linear Regression @Felix     %%%
%%%%%%%%%%%%%%%%%%%%%%%%%%%%%%%%%%%%%%%%%%%%%%%%%%%%%
\subsection{Linear Resgression}\label{sub:linReg}
% LINEAR REGRESSION %

%\Subsection{Linear Regression}\label{sub:linReg}

\begin{figure}
    \centering
    \includegraphics[width=0.5\textwidth]{scatter_plot_and_linear_regression_no_title.jpg}
    \label{scatter_plot}
\end{figure}


%%%%%%%%%%%%%%%%%%%%%%%%%%%%%%%%%%%%%%%%%%%%%%%%%%%%%
%%%     Subsection - CNN @Yuria                   %%%
%%%%%%%%%%%%%%%%%%%%%%%%%%%%%%%%%%%%%%%%%%%%%%%%%%%%%
\subsection{CNN}\label{sub:cnn}
% CNN %

%\subsection{CNN}\label{sub:cnn}

We set up a convolutional neural network with the preprocessed oct images as input.
The main idea is, that force acts continuously on the tissue and therefore its value depends on the past.
By considering several A-scans for force estimation at one timestamp, outliers are weakened and local connectivity between actual
measurement data and past data is created.

    \textit{Architecture} The image size of the oct is defined by the number of A-scans, whereby one A-scan contains data of 101 pixels.
To obtain a desired input image size $H$x$W$ for the model, the whole B-scan $H$x$W'$ is splitted by windowing it with a stride of 1.
Thus, a dimension of $D=W'-W-1$ of the input image arises.
The architecture of our CNN is adapted to ~paper gel force estimation~.
Due to the small amount of data, we use one convolutional layer followed by a rectified linear units layer and one 2x2 max pooling layer with a stride of two. 
The convolutional layer has 32 filter with size of 3x6 and a stride of one.

\begin{figure}
    \centering
    \includegraphics[width=0.9\textwidth]{cnn.png}
    \caption{Architecture of the CNN}
    \label{fig:cnn_archi}
\end{figure}



%%%%%%%%%%%%%%%%%%%%%%%%%%%%%%%%%%%%%%%%%%%%%%%%%%%%%
%%%     Subsection - RNN @Lakshamanan             %%%
%%%%%%%%%%%%%%%%%%%%%%%%%%%%%%%%%%%%%%%%%%%%%%%%%%%%%
\subsection{RNN}\label{sub:rnn}
% RNN %

%\Subsection{RNN}\label{sub:rnn}
      RNN model exhibit dynamic temporal behavior for a time sequence. Unlike feedforward neural networks, RNNs can use their internal state (memory) to process sequences of inputs. Since we are predicting the force acting at the needle, it would be good to know the previous history of the force sensor value for the OCT data in order to predict the force better. 


%%%%%%%%%%%%%%%%%%%%%%%%%%%%%%%%%%%%%%%%%%%%%%%%%%%%%
%%%     Section - Results                         %%%
%%%%%%%%%%%%%%%%%%%%%%%%%%%%%%%%%%%%%%%%%%%%%%%%%%%%%
\section{Results}\label{sec:results}
% RESULTS %

%\Section{Results}\label{sec:results}
The results of the different models are presented individually and compared by the mean squared error to emphasize 
large errors over small ones.

\subsection{Linear Regression}
\subsection{CNN}
%\subsection{RNN}
      RNN is trained with the extracted and preprocessed data feature acquired in the experiment (depth at maximum intensity) from the OCT depth scan as input to predict the force acting at the needle, as there exist a direct relationship between force data (measured using a force sensor) and the OCT image. This is a supervised learning, as force data from the force sensor is initially used for the training of RNN. Once the training is completed, one can predict the force acting at the needle with the OCT data alone without the help of force sensor.
      \begin{figure}
    \centering
    \includegraphics[width=0.9\textwidth]{RNN-toolbox.png}
    \caption{RNN toolbox in MATLAB showing the progress after 1000 iterations}
    \label{fig:RNN toolbox}
\end{figure}
      
      Here, RNN is trained using Neural Network toolbox in MATLAB. A recurrent net layer is created with one input layer, one output layer and 5 hidden layers and with the training function trainlm. As you can see in fig. the RNN is trained with the training function Levenberg-Marquardt and the performance is measured by mean squared error. Training is started with a mean squared error of 0.00515 and ended up with 0.00115 at the end of 1000 iterations. Validation is automatically taken care in the Neural network training toolbox itself. Once the training is done, the trained model is saved for testing. 
      
      RNN trained data is now used to predict the force value from the given OCT data. The input OCT data for testing is acquired from the experimental data and the output from the RNN is compared with the force sensor value and the performance is measured as root mean square and is summarized in the table below for different input conditions (linear & stepwise movement of needle against metal and Phantom). 
    
\begin{figure}
    \centering
    \includegraphics[width=0.9\textwidth]{metallinear8.png}
    \caption{Force prediction and actual force for the poking the needle against metal linearly}
    \label{fig:metallinear8}
\end{figure}


\begin{figure}
    \centering
    \includegraphics[width=0.9\textwidth]{metallinear11.png}
    \caption{Force prediction and actual force for the poking the needle against metal linearly}
    \label{fig:metallinear11}
\end{figure}

\begin{figure}
    \centering
    \includegraphics[width=0.9\textwidth]{metalstepwise12.png}
    \caption{Force prediction and actual force for the poking the needle against metal stepwise}
    \label{fig:metalstepwise12}
\end{figure}

\begin{figure}
    \centering
    \includegraphics[width=0.9\textwidth]{metalstepwise14.png}
    \caption{Force prediction and actual force for the poking the needle against metal stepwise}
    \label{fig:metalstepwise14}
\end{figure}

\begin{figure}
    \centering
    \includegraphics[width=0.9\textwidth]{p1m3.png}
    \caption{Force prediction and actual force for the poking the needle against phantom}
    \label{fig:Phantom}
\end{figure}
      
      
      
\subsection{Comparison}



%%% BIBLIOGRAPHY %%%
\begin{thebibliography}{4}


\end{thebibliography}

\end{document}
