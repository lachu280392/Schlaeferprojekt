%%% This is the template Gessert has send us.     %%%
%%% Look at it for further explanation and tips. %%%

\documentclass[runningheads,a4paper]{llncs}

\usepackage{amssymb}
\setcounter{tocdepth}{3}
\usepackage{graphicx}

\usepackage{url}
\urldef{\mailsa}\path|{felix.wege, yuria.konda, lakshamanan.chockalingam}@tuhh.de|    
\newcommand{\keywords}[1]{\par\addvspace\baselineskip
\noindent\keywordname\enspace\ignorespaces#1}

\begin{document}

\mainmatter  % start of an individual contribution

% first the title is needed
\title{Force Prediction using OCT Sensor Data 
by Linear Regression, CNN and RNN}

\author{Felix Wege \and Yuria Konda \and Lakshamanan Chockalingam}

\authorrunning{Force Estimation using OCT Sensor Data}
% (feature abused for this document to repeat the title also on left hand pages)

\institute{Intelligent Systems in Medicine, Institute of Medical Technology,\\
Am Schwarzenberg-Campus 3, 21073 Hamburg\\
\mailsa\\
}

\maketitle


\begin{abstract}
Content of this student project is to develop a model that gives 
information about the force acting at a needle tip while plunging it against 
a tissue.
The purpose is to give the operator some kind of force feedback
without measuring it explicitly.
Input of this model is an optical coherence tomography (OCT) signal
that is measured at the needle tip.
Three models which are linear regression, convotional neural network
and recursive neural network are considered.

\keywords{Machine learning, linear regression, CNN, RNN, optical coherence tomography}
\end{abstract}


\section{Introduction}


\section{Measurement Set Up}


\section{Theory}

Not sure if necessary. (Hopefully not?)

\section{Models}

\subsection{Linear Regression}

@Felix Wege

\subsection{CNN}

@Yuria Konda

\subsection{RNN}

@Lakshamanam

\section{Results}


%%% BIBLIOGRAPHY %%%
\begin{thebibliography}{4}


\end{thebibliography}

\end{document}
