%%% This is the template Gessert has send us.     %%%
%%% Look at it for further explanation and tips.  %%%
%%%%%%%%%%%%%%%%%%%%%%%%%%%%%%%%%%%%%%%%%%%%%%%%%%%%%

\documentclass[runningheads,a4paper]{llncs}

\usepackage{amssymb}
\setcounter{tocdepth}{3}
\usepackage{graphicx}
\usepackage{hyperref}
\usepackage{cleveref}

\usepackage{url}
\urldef{\mailsa}\path|{felix.wege, yuria.konda, lakshamanan.chockalingam}@tuhh.de|    
\newcommand{\keywords}[1]{\par\addvspace\baselineskip
\noindent\keywordname\enspace\ignorespaces#1}

\graphicspath{{figures/}}

%%%%%%%%%%%%%%%%%%%%%%%%%%%%%%%%%%%%%%%%%%%%%%%%%%%%%
\begin{document}

\mainmatter  % start of an individual contribution

% first the title is needed
\title{Force Prediction using OCT Sensor Data 
by Linear Regression, CNN and RNN}

\author{Felix Wege \and Yuria Konda \and Lakshamanan Chockalingam}

\authorrunning{Force Estimation using OCT Sensor Data}
% (feature abused for this document to repeat the title also on left hand pages)

\institute{Intelligent Systems in Medicine, Institute of Medical Technology,\\
Am Schwarzenberg-Campus 3, 21073 Hamburg\\
\mailsa\\
}

\maketitle


\begin{abstract}
Content of this student project is to develop a model that gives 
information about the force acting at a needle tip while plunging it against 
a tissue.
The purpose is to give the operator some kind of force feedback
without measuring it explicitly.
Input of this model is an optical coherence tomography (OCT) signal
that is measured at the needle tip.
Three models which are linear regression, convolutional neural network
and recursive neural network are considered and tested with real data.

\keywords{Machine learning, linear regression, CNN, RNN, optical coherence tomography}
\end{abstract}

%%%%%%%%%%%%%%%%%%%%%%%%%%%%%%%%%%%%%%%%%%%%%%%%%%%%%
%%%     Section - Introduction                    %%%	
%%%%%%%%%%%%%%%%%%%%%%%%%%%%%%%%%%%%%%%%%%%%%%%%%%%%%
\section{Introduction}\label{sec:intro}
% INTRODUCTION % 

%\section{Introduction}\label{sec:introduction}

In medical approaches OCT scans are used to image the uppermost layer of biological tissues non-invasively.
For the purpose of being able to investigate internal aeras of the body with this technique, researchers have developed 
OCT needles. 
One main reason for this attempt was that large surgical interventions can be avoided in many cases what 
results in smaller physical damages and faster healing proesses of the patients.



%%%%%%%%%%%%%%%%%%%%%%%%%%%%%%%%%%%%%%%%%%%%%%%%%%%%%
%%%     Section - Data Acquisition                %%%
%%%%%%%%%%%%%%%%%%%%%%%%%%%%%%%%%%%%%%%%%%%%%%%%%%%%%
\section{Data Acquisition}\label{sec:dataAcqui}
% DATA ACQUISITION%

%\Section{Data Acquisition}\label{sec:dataAcqui}

As stated in the introduction \ref{sec:intro}, models are build to estimate the force at the needle tip 
using the OCT data as input.

\subsection{Data}
To gather ground thruth data for modelling and supervised learning, a force sensor is 
integrated into the OCT system by placing it at the end of the needle.

*PICTURE*

Data is collected by poking the needle against a metal plate back and forth in linear or stepwise motions.
The OCT sensor detects the deformation of the transparent material at the tip of the needle that leads to a faster reflection of the light and thus
changes the depth of the maximal reflection in the B-scan.
By this, only frontal forces without any ditributing factors are measured.
The transparent material of the OCT needle deforms up to 0.35mm and one A-scan is represented by 512 pixels.
The acting forces are up to ...(???) Newton by only considering the force in needle direction. (z direction)

* PICTURES OF BOTH DATA*

\subsection{Preprocessing}

In oder to make use of both data sources, force sensor data is interpolated to obtain the same sampling frequency and start
and end points are synchronized.
The size of the OCT image is reduced to 50 pixels above and below the mean position of the maximum intensities due to
computational reasons. Consequently, the reflection of the repetitive light is neglected.

\subsection{Freature extraction}




%%%%%%%%%%%%%%%%%%%%%%%%%%%%%%%%%%%%%%%%%%%%%%%%%%%%%
%%%     Section - Models                          %%%
%%%%%%%%%%%%%%%%%%%%%%%%%%%%%%%%%%%%%%%%%%%%%%%%%%%%%
\section{Models}

Three models were trained and evaluated.
Firstly, simple linear regression was used to describe the relationship between the depth at maximum intensity and force.
Secondly, a CNN was employed with the advantage of not relying on the feature extraction.
The third model was an RNN, which, by having a 'memory', are able to model dynamic behaviour.

%%%%%%%%%%%%%%%%%%%%%%%%%%%%%%%%%%%%%%%%%%%%%%%%%%%%%
%%%     Subsection - Linear Regression @Felix     %%%
%%%%%%%%%%%%%%%%%%%%%%%%%%%%%%%%%%%%%%%%%%%%%%%%%%%%%
\subsection{Linear Resgression}\label{sub:linReg}
% LINEAR REGRESSION %

%\Subsection{Linear Regression}\label{sub:linReg}


%%%%%%%%%%%%%%%%%%%%%%%%%%%%%%%%%%%%%%%%%%%%%%%%%%%%%
%%%     Subsection - CNN @Yuria                   %%%
%%%%%%%%%%%%%%%%%%%%%%%%%%%%%%%%%%%%%%%%%%%%%%%%%%%%%
\subsection{CNN}\label{sub:cnn}
% CNN %

%\subsection{CNN}\label{sub:cnn}

We set up a convolutional neural network with the preprocessed oct image as input image.
The main idea is, that force acts continuously and therefore depends on the past.
By considering several A-scans for force estimation at one timestamp, outliers are weakened and local connectivity between actual
measurement data and past data is created.

    \textit{Architecture} The image size of the oct is defined by the number of A-scans, whereby one A-scan contains data of 101 pixels.
To obtain a desired input image size $HxW$ for the model, the whole B-scan $HxW'$ is splitted.
Thus, a dimension of $D = W' - W - 1$ of the input image arises.
The architecture of our CNN is adapted to ~paper gel force estimation~.
As represented in figure ~figure~, we use four convolutional layers, all followed by ReLu layers and 2x2 max pooling layers with a stride of two. 
C1, C2 and C3 with a filter size of 3 x 3 and a stride of one.




%%%%%%%%%%%%%%%%%%%%%%%%%%%%%%%%%%%%%%%%%%%%%%%%%%%%%
%%%     Subsection - RNN @Lakshamanan             %%%
%%%%%%%%%%%%%%%%%%%%%%%%%%%%%%%%%%%%%%%%%%%%%%%%%%%%%
\subsection{RNN}\label{sub:rnn}
% RNN %

%\Subsection{RNN}\label{sub:rnn}
      RNN model exhibit dynamic temporal behavior for a time sequence. Unlike feedforward neural networks, RNNs can use their internal state (memory) to process sequences of inputs. Since we are predicting the force acting at the needle, it would be good to know the previous history of the force sensor value for the OCT data in order to predict the force better. 


%%%%%%%%%%%%%%%%%%%%%%%%%%%%%%%%%%%%%%%%%%%%%%%%%%%%%
%%%     Section - Results                         %%%
%%%%%%%%%%%%%%%%%%%%%%%%%%%%%%%%%%%%%%%%%%%%%%%%%%%%%
\section{Results}\label{sec:results}
% RESULTS %

%\Section{Results}\label{sec:results}




%%% BIBLIOGRAPHY %%%
\begin{thebibliography}{4}


\end{thebibliography}

\end{document}
