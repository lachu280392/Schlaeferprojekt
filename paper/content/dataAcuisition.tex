% DATA ACQUISITION%

%\Section{Data Acquisition}\label{sec:dataAcqui}

As stated in the introduction \ref{sec:intro}, models are build to estimate the force at the needle tip 
using the OCT data as input.

\subsection{Data}
To gather ground thruth data for modelling and supervised learning, a force sensor is 
integrated into the OCT system by placing it at the end of the needle.

*PICTURE*

Data is collected by poking the needle against a metal plate back and forth in linear or stepwise motions.
The OCT sensor detects the deformation of the transparent material at the tip of the needle that leads to a faster reflection of the light and thus
changes the depth of the maximal reflection in the B-scan.
By this, only frontal forces without any ditributing factors are measured.
The transparent material of the OCT needle deforms up to 0.35mm and one A-scan is represented by 512 pixels.
The acting forces are up to ...(???) Newton by only considering the force in needle direction. (z direction)

\begin{figure}
\centering
\includegraphics[scale=.7]{/figures/
* PICTURES OF BOTH DATA*

\subsection{Preprocessing}

In oder to make use of both data sources, force sensor data is interpolated to obtain the same sampling frequency and start
and end points are synchronized.
The size of the OCT image is reduced to 50 pixels above and below the mean position of the maximum intensities due to
computational reasons. Consequently, the reflection of the repetitive light is neglected.

\subsection{Freature extraction}


