% DATA ACQUISITION%

%\Section{Data Acquisition}\label{sec:dataAcqui}

As stated in the introduction \ref{sec:intro}, models are build to estimate the force at the needle tip 
using the OCT data as input.
For optimization and supervised learning, ground truth data are necessary. For this, a force sensor is 
integrated into the OCT system by placing it at the end of the needle.

*PICTURE*

Data is collected by moving the needle forward to poke against a metal plate continuously monitored.
The transparent tip part of the needle deforms what leads to a faster reflection of the light in the OCT sensor and thus
changes the depth of the maximal reflection.
This set up only measures frontal forces and calibrates the system ue to the stiff 
